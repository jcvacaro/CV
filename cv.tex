\documentclass[11pt]{article}

\usepackage[brazil]{babel}   
\usepackage[T1]{fontenc}
\usepackage[latin1]{inputenc}  
\usepackage{times}
\usepackage{url}
\usepackage{graphicx}
\usepackage{jcvcurr}

\photoscale{.2}
\photo[l]{juliano}

\leftheader{
  Address: Alu�zio de Azevedo, 185\\
  ZIP: 91210-220\\
  Phone: +55~51~999808880\\
  Email: \url{jcvacaro@gmail.com}\\
  Linkedin: \url{juliano-vacaro}
}

\rightheader{
  RG: 3075201347\\
  CPF: 803891200-49\\
  Marital status: Single\\
  Age: 36
}

\title{Juliano Cardoso Vacaro}

\begin{document}

%--------------------------------------------------------------------------------

\makeheaders
\maketitle

\begin{cv}

%--------------------------------------------------------------------------------

\begin{cvlist}{PERSONAL PROFILE}

\item
*I'm visually impaired*, and that's my bad news. However, the good news is that I can overcome this limitation with technology by using accessibility software, a great passion for facing challenges, and seeking opportunities to learn every day. These values helped me achieve the highest degree at University, two R\&D Brazil awards at HP, as well as innovations in patents and technical publications at HPE. Doing things different, asking myself how to do better in a collaborative way is what I like the most. That's probably the reason I've been working with technologies such as Cloud Computing, Software Defined Networking, Security, and Machine Learning for the recent years.

\end{cvlist}

%--------------------------------------------------------------------------------

\begin{cvlist}{SPECIALTIES}

\item 
\begin{jcvlist2}

\item Comprehensive knowledge in networking: Software Defined Networking, Network Function Virtualization, 
      Network Access Control, secure communication schemes for distributed systems at scale;
\item Experience in security: cryptographic algorithms, Public Key Infrastructure including key management,
      digital certificates, and libraries;
\item Knowledge in Machine Learning: supervised and unsupervised techniques, feature engineering, 
      Deep Learning, Reinforcement Learning;
\item Recent experience in Blockchain (Ethereum) and related technologies such as IPFS 
      (Interplanetary FileSystem);)
\item Embedded technologies: FPGA, Windows CE/.NET Compact Framework;
\item Smart Card technology and related standards (Global Platform);
\item Software engineering skills: Design Patterns, Refactoring, Test Driven Development, 
      and Continuous Integration;
\item Project development based on C/C++, C\#, Java, Python, Go, and tools: MS Visual Studio, Eclipse, SVN/Git;
\item Advanced english: reading, writing, and speaking.

\end{jcvlist2}

\end{cvlist}

%--------------------------------------------------------------------------------%

\begin{cvlist}{EDUCATION}

\begin{institution}{2005--2007}{Universidade Federal do Rio Grande do Sul}{Porto Alegre - RS}
  \item M.S., Computer Science;
  \item Fault Tolerance in Distributed Systems.
\end{institution}

\begin{institution}{1999--2003}{Pontif�cia Universidade Cat�lica do Rio Grande do Sul}{Porto Alegre - RS}
  \item B.S., Computer Science;
  \item Graduated with the best performance rate.
\end{institution}

\end{cvlist}

%--------------------------------------------------------------------------------

\newpage

%--------------------------------------------------------------------------------

\begin{cvlist}{EXPERIENCE}

\begin{job}{Since 05/2018}{Hewlett Packard}{Porto Alegre - RS}{Senior Software Engineer/Researcher}
  \item Focused on Artificial Intelligence, Reinforcement Learning;
  \item Currently defining the architecture for the distributed training algorithm DeepMind IMPALA based on Linux containers;
\end{job}

\begin{job}{11/2007--05/2018}{Hewlett Packard Enterprise}{Porto Alegre - RS}{Technical Leader}
  \item Experience leading small/medium teams focused on product and research;
  \item Applying Machine Learning for network access control and visibility;
  \item Patent inventor of a new SDN-based load balancing technique for NFV;
  \item Worked on Cloud/DataCenter products focused on customer demands;
  \item Designed/Implemented the WebServer infrastructure for printing devices; 
\end{job}

\begin{job}{12/2006--11/2007}{ACTIA do Brasil}{Porto Alegre - RS}{Software Engineer}
  \item Implemented the remote diagnostic infrastructure for FIAT vehicles;
  \item Worked based on C++ wxWidgets.
\end{job}

\begin{job}{2002--2005}{FAURGS - Banrisul S.A.}{Porto Alegre - RS}{Developer}
  \item Worked in the security group, in the SmartCard technology team;
  \item Implemented the cryptographic support needed to issue smart cards in production through the IBM cryptographic board (ICSF).
\end{job}

\begin{job}{2000--2002}{Hardware Design Support Group (GAPH) - PUCRS}{Porto Alegre - RS}{Trainee}
  \item Implemented a IP soft core MAC Ethernet in the FPGA platform;
  \item Worked in hardware based projects using VHDL.
\end{job}

\end{cvlist}

%--------------------------------------------------------------------------------%

\begin{cvlist}{ACCOMPLISHMENTS}

\begin{institution}{Aug 2017}{Neural Networks and Deep Learning}{Coursera}
  \item Andrew Ng / deeplearning.ai
\end{institution}

\begin{institution}{Mar 2017}{Become a CompTIA Security+ Certified Security Professional}{LinkedIn}
  \item Mike Chapple
\end{institution}

\begin{institution}{Aug 2016}{High Performance Service Chaining for Ethernet Networks}{SIGCOMM}
  \item J. Vacaro, R. Eichelberger, S. Tandel, S. Banerjee, P. Bottorff, D. Fedyk
\end{institution}

\begin{institution}{Dec 2015}{Cloud Networking}{Coursera}
  \item P. Brighten Godfrey \& Ankit Singla / University of Illinois
\end{institution}

\begin{institution}{Jun 2015}{Server load balancing}{Patent}
  \item Juliano Vacaro, S�bastien Tandel, Bryan Stiekes
\end{institution}

\begin{institution}{Apr 2015}{Cloud Computing Concepts: Part 2}{Coursera}
  \item Indranil Gupta / University of Illinois
\end{institution}

\begin{institution}{Sep 2014}{Machine Learning}{Coursera}
  \item Andrew Ng
\end{institution}

\end{cvlist}

%--------------------------------------------------------------------------------%

\cvplace{Porto Alegre}
\date{September 2018}

%--------------------------------------------------------------------------------%

\end{cv}

\end{document}
